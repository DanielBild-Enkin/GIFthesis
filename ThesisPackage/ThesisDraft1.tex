%%%%%%%%%%%%%%%%%%%%%%%%%%%%%%%%%%%%%%%%%%%%%%%%%%%%%%%%%%%%%%%%%%%%%%
% Template for a UBC-compliant dissertation
% At the minimum, you will need to change the information found
% after the "Document meta-data"
%
%!TEX TS-program = pdflatex
%!TEX encoding = UTF-8 Unicode

%% The ubcdiss class provides several options:
%%   gpscopy (aka fogscopy)
%%       set parameters to exactly how GPS specifies
%%         * single-sided
%%         * page-numbering starts from title page
%%         * the lists of figures and tables have each entry prefixed
%%           with 'Figure' or 'Table'
%%       This can be tested by `\ifgpscopy ... \else ... \fi'
%%   10pt, 11pt, 12pt
%%       set default font size
%%   oneside, twoside
%%       whether to format for single-sided or double-sided printing
%%   balanced
%%       when double-sided, ensure page content is centred
%%       rather than slightly offset (the default)
%%   singlespacing, onehalfspacing, doublespacing
%%       set default inter-line text spacing; the ubcdiss class
%%       provides \textspacing to revert to this configured spacing
%%   draft
%%       disable more intensive processing, such as including
%%       graphics, etc.
%%

% For submission to GPS
\documentclass[gpscopy,onehalfspacing,11pt]{ubcdiss}

% For your own copies (looks nicer)
% \documentclass[balanced,twoside,11pt]{ubcdiss}

%%%%%%%%%%%%%%%%%%%%%%%%%%%%%%%%%%%%%%%%%%%%%%%%%%%%%%%%%%%%%%%%%%%%%%
%%%%%%%%%%%%%%%%%%%%%%%%%%%%%%%%%%%%%%%%%%%%%%%%%%%%%%%%%%%%%%%%%%%%%%
%%
%% FONTS:
%% 
%% The defaults below configures Times Roman for the serif font,
%% Helvetica for the sans serif font, and Courier for the
%% typewriter-style font.  Configuring fonts can be time
%% consuming; we recommend skipping to END FONTS!
%% 
%% If you're feeling brave, have lots of time, and wish to use one
%% your platform's native fonts, see the commented out bits below for
%% XeTeX/XeLaTeX.  This is not for the faint at heart. 
%% (And shouldn't you be writing? :-)
%%

%% NFSS font specification (New Font Selection Scheme)
\usepackage{times,mathptmx,courier}
\usepackage[scaled=.92]{helvet}

%% Math or theory people may want to include the handy AMS macros
\usepackage{amssymb}
\usepackage{amsmath}
\usepackage{amsfonts}

%% The pifont package provides access to the elements in the dingbat font.   
%% Use \ding{##} for a particular dingbat (see p7 of psnfss2e.pdf)
%%   Useful:
%%     51,52 different forms of a checkmark
%%     54,55,56 different forms of a cross (saltyre)
%%     172-181 are 1-10 in open circle (serif)
%%     182-191 are 1-10 black circle (serif)
%%     192-201 are 1-10 in open circle (sans serif)
%%     202-211 are 1-10 in black circle (sans serif)
%% \begin{dinglist}{##}\item... or dingautolist (which auto-increments)
%% to create a bullet list with the provided character.
\usepackage{pifont}

%%%%%%%%%%%%%%%%%%%%%%%%%%%%%%%%%%%%%%%%%%%%%%%%%%%%%%%%%%%%%%%%%%%%%%
%% Configure fonts for XeTeX / XeLaTeX using the fontspec package.
%% Be sure to check out the fontspec documentation.
%\usepackage{fontspec,xltxtra,xunicode}	% required
%\defaultfontfeatures{Mapping=tex-text}	% recommended
%% Minion Pro and Myriad Pro are shipped with some versions of
%% Adobe Reader.  Adobe representatives have commented that these
%% fonts can be used outside of Adobe Reader.
%\setromanfont[Numbers=OldStyle]{Minion Pro}
%\setsansfont[Numbers=OldStyle,Scale=MatchLowercase]{Myriad Pro}
%\setmonofont[Scale=MatchLowercase]{Andale Mono}

%% Other alternatives:
%\setromanfont[Mapping=tex-text]{Adobe Caslon}
%\setsansfont[Scale=MatchLowercase]{Gill Sans}
%\setsansfont[Scale=MatchLowercase,Mapping=tex-text]{Futura}
%\setmonofont[Scale=MatchLowercase]{Andale Mono}
%\newfontfamily{\SYM}[Scale=0.9]{Zapf Dingbats}
%% END FONTS
%%%%%%%%%%%%%%%%%%%%%%%%%%%%%%%%%%%%%%%%%%%%%%%%%%%%%%%%%%%%%%%%%%%%%%
%%%%%%%%%%%%%%%%%%%%%%%%%%%%%%%%%%%%%%%%%%%%%%%%%%%%%%%%%%%%%%%%%%%%%%



%%%%%%%%%%%%%%%%%%%%%%%%%%%%%%%%%%%%%%%%%%%%%%%%%%%%%%%%%%%%%%%%%%%%%%
%%%%%%%%%%%%%%%%%%%%%%%%%%%%%%%%%%%%%%%%%%%%%%%%%%%%%%%%%%%%%%%%%%%%%%
%%
%% Recommended packages
%%
\usepackage{checkend}	% better error messages on left-open environments
\usepackage{graphicx}	% for incorporating external images

%% booktabs: provides some special commands for typesetting tables as used
%% in excellent journals.  Ignore the examples in the Lamport book!
\usepackage{booktabs}

%% listings: useful support for including source code listings, with
%% optional special keyword formatting.  The \lstset{} causes
%% the text to be typeset in a smaller sans serif font, with
%% proportional spacing.
\usepackage{listings}
\lstset{basicstyle=\sffamily\scriptsize,showstringspaces=false,fontadjust}

%% The acronym package provides support for defining acronyms, providing
%% their expansion when first used, and building glossaries.  See the
%% example in glossary.tex and the example usage throughout the example
%% document.
%% NOTE: to use \MakeTextLowercase in the \acsfont command below,
%%   we *must* use the `nohyperlinks' option -- it causes errors with
%%   hyperref otherwise.  See Section 5.2 in the ``LaTeX 2e for Class
%%   and Package Writers Guide'' (clsguide.pdf) for details.
\usepackage[printonlyused,nohyperlinks]{acronym}
%% The ubcdiss.cls loads the `textcase' package which provides commands
%% for upper-casing and lower-casing text.  The following causes
%% the acronym package to typeset acronyms in small-caps
%% as recommended by Bringhurst.
\renewcommand{\acsfont}[1]{{\scshape \MakeTextLowercase{#1}}}

%% color: add support for expressing colour models.  Grey can be used
%% to great effect to emphasize other parts of a graphic or text.
%% For an excellent set of examples, see Tufte's "Visual Display of
%% Quantitative Information" or "Envisioning Information".
\usepackage{color}
\definecolor{greytext}{gray}{0.5}

%% comment: provides a new {comment} environment: all text inside the
%% environment is ignored.
%%   \begin{comment} ignored text ... \end{comment}
\usepackage{comment}

%% The natbib package provides more sophisticated citing commands
%% such as \citeauthor{} to provide the author names of a work,
%% \citet{} to produce an author-and-reference citation,
%% \citep{} to produce a parenthetical citation.
%% We use \citeeg{} to provide examples
\usepackage[numbers,sort&compress]{natbib}
\newcommand{\citeeg}[1]{\citep[e.g.,][]{#1}}

%% The titlesec package provides commands to vary how chapter and
%% section titles are typeset.  The following uses more compact
%% spacings above and below the title.  The titleformat that follow
%% ensure chapter/section titles are set in singlespace.
\usepackage[compact]{titlesec}
\titleformat*{\section}{\singlespacing\raggedright\bfseries\Large}
\titleformat*{\subsection}{\singlespacing\raggedright\bfseries\large}
\titleformat*{\subsubsection}{\singlespacing\raggedright\bfseries}
\titleformat*{\paragraph}{\singlespacing\raggedright\itshape}

%% The caption package provides support for varying how table and
%% figure captions are typeset.
\usepackage[format=hang,indention=-1cm,labelfont={bf},margin=1em]{caption}

%% url: for typesetting URLs and smart(er) hyphenation.
%% \url{http://...} 
\usepackage{url}
\urlstyle{sf}	% typeset urls in sans-serif


%%%%%%%%%%%%%%%%%%%%%%%%%%%%%%%%%%%%%%%%%%%%%%%%%%%%%%%%%%%%%%%%%%%%%%
%%%%%%%%%%%%%%%%%%%%%%%%%%%%%%%%%%%%%%%%%%%%%%%%%%%%%%%%%%%%%%%%%%%%%%
%%
%% Possibly useful packages: you may need to explicitly install
%% these from CTAN if they aren't part of your distribution;
%% teTeX seems to ship with a smaller base than MikTeX and MacTeX.
%%
%\usepackage{pdfpages}	% insert pages from other PDF files
%\usepackage{longtable}	% provide tables spanning multiple pages
%\usepackage{chngpage}	% support changing the page widths on demand
%\usepackage{tabularx}	% an enhanced tabular environment

%% enumitem: support pausing and resuming enumerate environments.
%\usepackage{enumitem}

%% rotating: provides two environments, sidewaystable and sidewaysfigure,
%% for typesetting tables and figures in landscape mode.  
%\usepackage{rotating}

%% subfig: provides for including subfigures within a figure,
%% and includes being able to separately reference the subfigures.
%\usepackage{subfig}

%% ragged2e: provides several new new commands \Centering, \RaggedLeft,
%% \RaggedRight and \justifying and new environments Center, FlushLeft,
%% FlushRight and justify, which set ragged text and are easily
%% configurable to allow hyphenation.
%\usepackage{ragged2e}

%% The ulem package provides a \sout{} for striking out text and
%% \xout for crossing out text.  The normalem and normalbf are
%% necessary as the package messes with the emphasis and bold fonts
%% otherwise.
%\usepackage[normalem,normalbf]{ulem}    % for \sout

%%%%%%%%%%%%%%%%%%%%%%%%%%%%%%%%%%%%%%%%%%%%%%%%%%%%%%%%%%%%%%%%%%%%%%
%% HYPERREF:
%% The hyperref package provides for embedding hyperlinks into your
%% document.  By default the table of contents, references, citations,
%% and footnotes are hyperlinked.
%%
%% Hyperref provides a very handy command for doing cross-references:
%% \autoref{}.  This is similar to \ref{} and \pageref{} except that
%% it automagically puts in the *type* of reference.  For example,
%% referencing a figure's label will put the text `Figure 3.4'.
%% And the text will be hyperlinked to the appropriate place in the
%% document.
%%
%% Generally hyperref should appear after most other packages

%% The following puts hyperlinks in very faint grey boxes.
%% The `pagebackref' causes the references in the bibliography to have
%% back-references to the citing page; `backref' puts the citing section
%% number.  See further below for other examples of using hyperref.
%% 2009/12/09: now use `linktocpage' (Jacek Kisynski): GPS now prefers
%%   that the ToC, LoF, LoT place the hyperlink on the page number,
%%   rather than the entry text.
\usepackage[bookmarks,bookmarksnumbered,%
    allbordercolors={0.8 0.8 0.8},%
    pagebackref,linktocpage%
    ]{hyperref}
%% The following change how the the back-references text is typeset in a
%% bibliography when `backref' or `pagebackref' are used
\renewcommand\backrefpagesname{\(\rightarrow\) pages}
\renewcommand\backref{\textcolor{greytext} \backrefpagesname\ }

%% The following uses most defaults, which causes hyperlinks to be
%% surrounded by colourful boxes; the colours are only visible in
%% PDFs and don't show up when printed:
%\usepackage[bookmarks,bookmarksnumbered]{hyperref}

%% The following disables the colourful boxes around hyperlinks.
%\usepackage[bookmarks,bookmarksnumbered,pdfborder={0 0 0}]{hyperref}

%% The following disables all hyperlinking, but still enabled use of
%% \autoref{}
%\usepackage[draft]{hyperref}

%% The following commands causes chapter and section references to
%% uppercase the part name.
\renewcommand{\chapterautorefname}{Chapter}
\renewcommand{\sectionautorefname}{Section}
\renewcommand{\subsectionautorefname}{Section}
\renewcommand{\subsubsectionautorefname}{Section}

%% If you have long page numbers (e.g., roman numbers in the 
%% preliminary pages for page 28 = xxviii), you might need to
%% uncomment the following and tweak the \@pnumwidth length
%% (default: 1.55em).  See the tocloft documentation at
%% http://www.ctan.org/tex-archive/macros/latex/contrib/tocloft/
% \makeatletter
% \renewcommand{\@pnumwidth}{3em}
% \makeatother

%%%%%%%%%%%%%%%%%%%%%%%%%%%%%%%%%%%%%%%%%%%%%%%%%%%%%%%%%%%%%%%%%%%%%%
%%%%%%%%%%%%%%%%%%%%%%%%%%%%%%%%%%%%%%%%%%%%%%%%%%%%%%%%%%%%%%%%%%%%%%
%%
%% Some special settings that controls how text is typeset
%%
% \raggedbottom		% pages don't have to line up nicely on the last line
% \sloppy		% be a bit more relaxed in inter-word spacing
% \clubpenalty=10000	% try harder to avoid orphans
% \widowpenalty=10000	% try harder to avoid widows
% \tolerance=1000

%% And include some of our own useful macros
% This file provides examples of some useful macros for typesetting
% dissertations.  None of the macros defined here are necessary beyond
% for the template documentation, so feel free to change, remove, and add
% your own definitions.
%
% We recommend that you define macros to separate the semantics
% of the things you write from how they are presented.  For example,
% you'll see definitions below for a macro \file{}: by using
% \file{} consistently in the text, we can change how filenames
% are typeset simply by changing the definition of \file{} in
% this file.
% 
%% The following is a directive for TeXShop to indicate the main file
%%!TEX root = diss.tex

\newcommand{\NA}{\textsc{n/a}}	% for "not applicable"
\newcommand{\eg}{e.g.,\ }	% proper form of examples (\eg a, b, c)
\newcommand{\ie}{i.e.,\ }	% proper form for that is (\ie a, b, c)
\newcommand{\etal}{\emph{et al}}

% Some useful macros for typesetting terms.
\newcommand{\file}[1]{\texttt{#1}}
\newcommand{\class}[1]{\texttt{#1}}
\newcommand{\latexpackage}[1]{\href{http://www.ctan.org/macros/latex/contrib/#1}{\texttt{#1}}}
\newcommand{\latexmiscpackage}[1]{\href{http://www.ctan.org/macros/latex/contrib/misc/#1.sty}{\texttt{#1}}}
\newcommand{\env}[1]{\texttt{#1}}
\newcommand{\BibTeX}{Bib\TeX}

% Define a command \doi{} to typeset a digital object identifier (DOI).
% Note: if the following definition raise an error, then you likely
% have an ancient version of url.sty.  Either find a more recent version
% (3.1 or later work fine) and simply copy it into this directory,  or
% comment out the following two lines and uncomment the third.
\DeclareUrlCommand\DOI{}
\newcommand{\doi}[1]{\href{http://dx.doi.org/#1}{\DOI{doi:#1}}}
%\newcommand{\doi}[1]{\href{http://dx.doi.org/#1}{doi:#1}}

% Useful macro to reference an online document with a hyperlink
% as well with the URL explicitly listed in a footnote
% #1: the URL
% #2: the anchoring text
\newcommand{\webref}[2]{\href{#1}{#2}\footnote{\url{#1}}}

% epigraph is a nice environment for typesetting quotations
\makeatletter
\newenvironment{epigraph}{%
	\begin{flushright}
	\begin{minipage}{\columnwidth-0.75in}
	\begin{flushright}
	\@ifundefined{singlespacing}{}{\singlespacing}%
    }{
	\end{flushright}
	\end{minipage}
	\end{flushright}}
\makeatother

% \FIXME{} is a useful macro for noting things needing to be changed.
% The following definition will also output a warning to the console
\newcommand{\FIXME}[1]{\typeout{**FIXME** #1}\textbf{[FIXME: #1]}}

% END


%%%%%%%%%%%%%%%%%%%%%%%%%%%%%%%%%%%%%%%%%%%%%%%%%%%%%%%%%%%%%%%%%%%%%%
%%%%%%%%%%%%%%%%%%%%%%%%%%%%%%%%%%%%%%%%%%%%%%%%%%%%%%%%%%%%%%%%%%%%%%
%%
%% Document meta-data: be sure to also change the \hypersetup information
%%

\title{On the Use of the \texttt{ubcdiss} Template}
%\subtitle{If you want a subtitle}

\author{Johnny Canuck}
\previousdegree{B. Basket Weaving, University of Illustrious Arts, 1991}
\previousdegree{M. Silly Walks, Another University, 1994}

% What is this dissertation for?
\degreetitle{Doctor of Philosophy}

\institution{The University of British Columbia}
\campus{Vancouver}

\faculty{The Faculty of XXX}
\department{Basket Weaving}
\submissionmonth{April}
\submissionyear{2192}

%% hyperref package provides support for embedding meta-data in .PDF
%% files
\hypersetup{
  pdftitle={Change this title!  (DRAFT: \today)},
  pdfauthor={Johnny Canuck},
  pdfkeywords={Your keywords here}
}

%%%%%%%%%%%%%%%%%%%%%%%%%%%%%%%%%%%%%%%%%%%%%%%%%%%%%%%%%%%%%%%%%%%%%%
%%%%%%%%%%%%%%%%%%%%%%%%%%%%%%%%%%%%%%%%%%%%%%%%%%%%%%%%%%%%%%%%%%%%%%
%% 
%% The document content
%%

%% LaTeX's \includeonly commands causes any uses of \include{} to only
%% include files that are in the list.  This is helpful to produce
%% subsets of your thesis (e.g., for committee members who want to see
%% the dissertation chapter by chapter).  It also saves time by 
%% avoiding reprocessing the entire file.
%\includeonly{intro,conclusions}
%\includeonly{discussion}

\begin{document}

%%%%%%%%%%%%%%%%%%%%%%%%%%%%%%%%%%%%%%%%%%%%%%%%%%
%% From Thesis Components: Tradtional Thesis
%% <http://www.grad.ubc.ca/current-students/dissertation-thesis-preparation/order-components>

% Preliminary Pages (numbered in lower case Roman numerals)
%    1. Title page (mandatory)
\maketitle

%    2. Abstract (mandatory - maximum 350 words)
%% The following is a directive for TeXShop to indicate the main file
%%!TEX root = diss.tex

\chapter{Abstract}

This document provides brief instructions for using the \class{ubcdiss}
class to write a \acs{UBC}-conformant dissertation in \LaTeX.  This
document is itself written using the \class{ubcdiss} class and is
intended to serve as an example of writing a dissertation in \LaTeX.
This document has embedded \acp{URL} and is intended to be viewed
using a computer-based \ac{PDF} reader.

Note: Abstracts should generally try to avoid using acronyms.

Note: at \ac{UBC}, both the \ac{GPS} Ph.D. defence programme and the
Library's online submission system restricts abstracts to 350
words.

% Consider placing version information if you circulate multiple drafts
%\vfill
%\begin{center}
%\begin{sf}
%\fbox{Revision: \today}
%\end{sf}
%\end{center}

\cleardoublepage

%    3. Preface
%% The following is a directive for TeXShop to indicate the main file
%%!TEX root = diss.tex

\chapter{Preface}


\cleardoublepage

%    4. Table of contents (mandatory - list all items in the preliminary pages
%    starting with the abstract, followed by chapter headings and
%    subheadings, bibliographies and appendices)
\tableofcontents
\cleardoublepage	% required by tocloft package

%    5. List of tables (mandatory if thesis has tables)
\listoftables
\cleardoublepage	% required by tocloft package

%    6. List of figures (mandatory if thesis has figures)
\listoffigures
\cleardoublepage	% required by tocloft package

%    7. List of illustrations (mandatory if thesis has illustrations)
%    8. Lists of symbols, abbreviations or other (optional)

%    9. Glossary (optional)
%% The following is a directive for TeXShop to indicate the main file
%%!TEX root = diss.tex

\chapter{Glossary}

This glossary uses the handy \latexpackage{acroynym} package to automatically
maintain the glossary.  It uses the package's \texttt{printonlyused}
option to include only those acronyms explicitly referenced in the
\LaTeX\ source.

% use \acrodef to define an acronym, but no listing
\acrodef{UI}{user interface}
\acrodef{UBC}{University of British Columbia}

% The acronym environment will typeset only those acronyms that were
% *actually used* in the course of the document
\begin{acronym}[ANOVA]
\acro{ANOVA}[ANOVA]{Analysis of Variance\acroextra{, a set of
  statistical techniques to identify sources of variability between groups}}
\acro{API}{application programming interface}
\acro{CTAN}{\acroextra{The }Common \TeX\ Archive Network}
\acro{DOI}{Document Object Identifier\acroextra{ (see
    \url{http://doi.org})}}
\acro{GPS}[GPS]{Graduate and Postdoctoral Studies}
\acro{PDF}{Portable Document Format}
\acro{RCS}[RCS]{Revision control system\acroextra{, a software
    tool for tracking changes to a set of files}}
\acro{TLX}[TLX]{Task Load Index\acroextra{, an instrument for gauging
  the subjective mental workload experienced by a human in performing
  a task}}
\acro{UML}{Unified Modelling Language\acroextra{, a visual language
    for modelling the structure of software artefacts}}
\acro{URL}{Unique Resource Locator\acroextra{, used to describe a
    means for obtaining some resource on the world wide web}}
\acro{W3C}[W3C]{\acroextra{the }World Wide Web Consortium\acroextra{,
    the standards body for web technologies}}
\acro{XML}{Extensible Markup Language}
\end{acronym}

% You can also use \newacro{}{} to only define acronyms
% but without explictly creating a glossary
% 
% \newacro{ANOVA}[ANOVA]{Analysis of Variance\acroextra{, a set of
%   statistical techniques to identify sources of variability between groups.}}
% \newacro{API}[API]{application programming interface}
% \newacro{GOMS}[GOMS]{Goals, Operators, Methods, and Selection\acroextra{,
%   a framework for usability analysis.}}
% \newacro{TLX}[TLX]{Task Load Index\acroextra{, an instrument for gauging
%   the subjective mental workload experienced by a human in performing
%   a task.}}
% \newacro{UI}[UI]{user interface}
% \newacro{UML}[UML]{Unified Modelling Language}
% \newacro{W3C}[W3C]{World Wide Web Consortium}
% \newacro{XML}[XML]{Extensible Markup Language}
	% always input, since other macros may rely on it

\textspacing		% begin one-half or double spacing

%   10. Acknowledgements (optional)
%% The following is a directive for TeXShop to indicate the main file
%%!TEX root = diss.tex

\chapter{Acknowledgments}

Thank those people who helped you. 

Don't forget your parents or loved ones.

You may wish to acknowledge your funding sources.


%   11. Dedication (optional)

% Body of Thesis (not all sections may apply)
\mainmatter

\acresetall	% reset all acronyms used so far

%    1. Introduction
%% The following is a directive for TeXShop to indicate the main file
%%!TEX root = diss.tex

\chapter{Introduction}
\label{ch:Introduction}

\begin{epigraph}
\emph{The 'true' is only the expedient in our way of thinking, just as the 'right' is only the expedient in our way of behaving.}\\
---~William James (1909)
\end{epigraph}


In all cases these are first guesses at what needs to be in each section more or less detail need to be added.


\section{What problems}
\label{sec:What problems}

Geophysical inversions, specifically potential fields 
include formulation of non-regularized inverse problem

\section{Difficulties with said problems }
\label{sec:Difficulties with said problems }

The standard way to fit a set of parameters to a set of data (especially when they are related by a linear operator) is least squares optimization. This is rendered problematic since, in general, geo-physical inversions are ill-conditioned (define) and undetermined (define) (\cite{oldenburg2005inversion} other sources I'm sure). In specific potential fields are particularly under-determined due to the lack of any depth information in the data.

show some form of problems with forward operator matrix in PF inversion

\section{Solutions to said difficulties}
\label{sec:Solutions to said difficulties}

To mitigate the difficulties presented above an extra term is added to the optimization. 

\begin{align}
\phi = \phi_d + \beta\phi_m
\end{align}
\label{eq:objective function}

where $\phi_m$ is called the \ac{MOF} or model norm. This $\phi_m$ can be defined in many ways, following  \cite{oldenburg2005inversion}

\begin{align}
\phi_m(m) &= \alpha_s\int(m-m_{ref})^2dx+\alpha_x\int\bigg(\frac{d}{dx}(m-m_{ref})\bigg)^2dx\\
&=\alpha_s\|\textbf{W}_s(m-m_{ref})\|^2_2+\alpha_x\|\textbf{W}_x(m-m_{ref})\|^2_2
\end{align}
\label{eq:MOF}

in higher dimensions more smoothness terms can be added. The $\textbf{W}$ terms contain both the operator (identity for $\textbf{W}_s$ and derivative for $\textbf{W}_x$ and other dimensions) and the relative weight each cell or face contributes to the \ac{MOF}. This gives us several levers to add a-priori information into the inversion.

The \ac{MOF} allows us to mathematically solve the problem by adding a priori information into the inversion. Namely we assume that the recovered model should be small and smooth. There are times when this is desired but often we have more specific information about the true model that needs to be inserted into the inversion. Luckily the various terms in the \ac{MOF} allow us to add a significant amount of information is various ways to the inversion.
	It must be said that all of the techniques listed below are not novel. Many researchers before me have used exactly these techniques to constrain inversions (\cite{williams2008geologicall},\cite{Lelievre2009Integrating} among other (still need to add more)). What is novel in this thesis is the creation of a suit of tools (created by me and the rest of the GIF group) to make the incorporation of geological data into the \ac{MOF} of inversions easy even in non-trivial cases.

\subsection{$\alpha$ coefficients}
\label{sec:alpha coefficients}

broad strokes weights the relative importance of the smallness and smoothness in the various directions. can also be thought of as length scales

\subsection{Reference Models}
\label{sec:Reference Models}

we don't always want a model to be close to zero. Sometimes it should be close to another constant sometimes we have guesses of the property in some places and want the inversion result to be close to that value

\subsection{Weighting matrices}
\label{sec:Weighting matrices}

much more precise. Can put interfaces in precise locations. Can also force a model towards the reference model where we are more sure
\\\\
along with the terms in the \ac{MOF} other parts of the optimization algorithm (may need more info in the optimization) can be used to add information into the inversion

\subsection{Initial Model}
\label{sec:Initial Model}

In the optimization we assume that the initial guess is near enough to the truth that the problem is locally convex. The initial model is important in that way. In an under determined system it also provides a way to push the inversion towards a given result. Often the initial model is simply the reference model, or the reference model shifted slightly to keep it within the bounds

\subsection{Bounds}
\label{sec:Bounds}

we can also set values that each cell of the final model must lie between. This allows for a hard setting of confidence intervals in the physical properties

\subsection{$L_p L_q$ weights}
\label{sec:Lp Lq weights}

Finally we can generalize the \ac{MOF} somewhat. In \autoref{eq:MOF} we used $L_2$ norms as this is a natural norm that promotes a smoothly varying model that is close to the reference model. We do not always want such a model and can change the norm used in the \ac{MOF}. Lower norms promote more sparsity in whatever measure they are being applied to. This leads to models being more compact (should lowever norms be applied to the smallness term) or more blocky with greater discontinuities (should lower norms be applied to one or more smoothness terms). Non $L_2$ norms can be applied across the whole \ac{MOF} or can be applied variably across the model. This allows for placing discontinuities in a given direction but not perfectly placing the location allowing the inversion algorithm more freedom to chose the location itself.

\section{Forms of A Priori Information}
\label{sec:Forms of A Priori Information}

\subsection{Bore Hole Data and the Use of Koenigsberger Ratios to Correct Bore Hole Susceptibility Measurements}
\label{sec: Bore Hole Data}

Bore holes provide physical property measurements at depth either by sending geophysical instruments down hole or by recovering a core and then measuring it subsequently in the lab. Bore holes can also provide qualitative rock unit information. Much work has been done on including physical property bore hole information(\cite{williams2008geologically} among others I'm sure). If we can convert the lithology information from bore holes into physical property information (using petrophysical measurements of reasonably similar rocks) we can use the information in the same fashion as a physical property bore hole logs.

In some contexts one physical property can be used as a proxy for others. In one case study, we have magnetic susceptibility measurements down hole but surface samples have been measured which reveal high magnetic remanence. The simple susceptibility measurement is drastically lower than the recovered effective susceptibility derived from the inversion of a magnetic survey over the area. This is due to the fact that the inversion is recovering effective susceptibility (induced magnetization plus \ac{NRM} normalized by and assuming the direction of earth's field in the location). To describe the method we need to discuss the general derivation of the magnetostatic problem and then discuss the effects of \ac{NRM} on measured data and the recovered inversion result. 

The following derivation follows the one in \cite{fournier2015cooperative}. We can derive the magnetostatic problem from Maxwell's equations. If we assume no free currents and no time varying electric field Maxwell's equations simplify to 

\begin{equation} 
\label{eq:maxwellB}
\mathbf{\nabla} \cdot \mathbf{B} = 0\\
\end{equation}
\begin{equation}
\label{eq:maxwelH}
\mathbf{\nabla} \times \mathbf{H} = 0\\   
\end{equation}
\begin{equation}
\label{eq:maxwellHB}
\mathbf{B} = \mu\mathbf{H},
\end{equation}
where $\mathbf{B}$ is the magnetic flux density measured in Tesla (T), $\mathbf{H}$ is the magnetic field measured in amperes per meter (A/m), and $\mu$ is magnetic permeability which relates  $\mathbf{B}$ to $\mathbf{H}$ in matter. We can rewrite $\mu$ to take the permeability of free space ($\mu_0$) into account and state that
\begin{equation}\label{eq:susc}
\mu = \mu_0(1 + \kappa),
\end{equation}
where $\mu_0$ is the permeability of free space ($4\pi\times10^{-7}\frac{Tm}{A}$), and $\kappa$ is the magnetic susceptibility of a material. $\kappa$ is dimensionless and describes the ability of a material to become magnetized under some field $\mathbf{H}$. The definition of $\kappa$ in \autoref{eq:susc}  gives us a definition for induced magnetization
\begin{equation}\label{eq:Mi}
\mathbf M_I = \kappa\textbf{H}.
\end{equation}
Since there are no free currents and \autoref{eq:maxwelH} states that $\mathbf{H}$ has no curl, it can be written as the gradient of a potential field
\begin{equation}\label{eq:phi}
\mathbf{H} = \mathbf{\nabla}\phi.
\end{equation}
Since, by \autoref{eq:maxwellB}, we assume that there are no magnetic monopoles, we approximate $\phi$ in terms of a dipole moment $\mathbf m$. If we have a magnetic dipole with a moment of $\mathbf m$ at a location $\mathbf r_Q$, then the potential field $\phi$ as measured at some $\mathbf r_P$ is given by
\begin{equation}\label{eq:phiOfmDiscrete}
\phi(r) = \frac{1}{4\pi}\mathbf m\cdot \mathbf{\nabla}\Big(\frac{1}{r}\Big),
\end{equation}
where
\begin{equation}\label{eq:rDef}
\mathbf r = \| \mathbf r_Q - \mathbf r_P\|_2.
\end{equation}
We can generalize \autoref{eq:phiOfmDiscrete} to a continuous form by replacing the discrete $\mathbf m$ with a continuous $\mathbf M$ and integrating
\begin{equation}\label{eq:phiOfmCont}
\phi( r) = \frac{1}{4\pi}\int_V\mathbf M\cdot \mathbf{\nabla}\Big(\frac{1}{r}\Big)dv.
\end{equation}
If we take the gradient of \autoref{eq:phiOfmCont} we find $\mathbf B$ the magnetic flux density,\\\\
\begin{equation}\label{eq:BOfPcont}
\mathbf B (\mathbf r_P) = \frac{1}{4\pi}\int_V\mathbf M\cdot \mathbf{\nabla}\mathbf{\nabla}\Big(\frac{1}{r}\Big)dv.
\end{equation}
In \autoref{eq:BOfPcont} the dependency of $\mathbf B$ on $\mathbf r_P$ is due to the fact that $r$ depends on $\mathbf r_P$ as in \autoref{eq:rDef}. In most geophysical surveys, the full vector $\mathbf B$ is not collected, usually only its magnitude, or the \ac{TMI},
\begin{equation}\label{eq:TMI}
B_{TMI} = \| \mathbf B_0 + \mathbf B_A\|_2
\end{equation}
where $\mathbf B_0$ is the primary field (earth's field) and $\mathbf B_A$ is the anomalous local field due to magnetization in the ground. For the purposes of geophysical exploration we are only interested in $\mathbf B_A$. Since we are only interested in $\mathbf B_A$ a useful quantity is therefor the \ac{TMA}, defined as
\begin{equation}\label{eq:TMA}
B_{TMA} = \| \mathbf B_{Total} + \mathbf B_0\|_2
\end{equation}
\ac{TMA} is difficult to measure directly but can be approximated assuming $\frac{\|\mathbf B_A\|}{\|\mathbf B_0\|} \ll 1$,
\begin{align}\label{eq:TMAaprox}
B_{TMA} &\simeq \mathbf B_A\cdot \hat{\mathbf B}_0\\
&=\|\mathbf B_0+\mathbf B_A\| - \|\mathbf B_0\|
\end{align}
We now have a formulation of $\mathbf B$ that depends on magnetization $\mathbf M$, the vector field of magnetization in the ground and the quantities collected in geophysical magnetics surveys. We will now take a more specific look at magnetization and its effects both on $\mathbf B$ and $B_{TMA}$. If we assume no self-demagnetization (which is reasonable for susceptibilities below $1\time10^{-2}$ \cite{lelievre2006magnetic}) the inducing magnetic field is constant over the volume to be inverted and total magnetization can be characterized by the following
\begin{align} \label{eq:magnetization}
\textbf{M} &= \mathbf M_I + \textbf{M}_{NRM}\\
\textbf{M} &= \kappa\textbf{H} + \textbf{M}_{NRM}
\end{align}
Here $\textbf M$ is the total magnetization, $\mathbf M_I$ is the induced magnetization as in \autoref{eq:Mi} , $\kappa$ is susceptibility as defined in \autoref{eq:susc}, $\textbf{H}$ is the inducing field (in this case of the geomagnetic field) and $\textbf{M}_{NRM}$ is the \ac{NRM}. $\textbf{M}_{NRM}$  is also characterized by what is called Koenigsberger ratio
\begin{equation} \label{eq:Koenigsberger}
Q = \frac{\textbf{M}_{NRM}}{ \kappa\textbf{H}} = \frac{\text{remanent magnetization}}{\text{induced magnetization}}
\end{equation}
In the case where \ac{NRM} is negligible, the direction of $\mathbf M$ isthe same as $\mathbf H$, meaning that $\mathbf B_A$ and $\mathbf B_0$ are in the same direction and making the approximation in \autoref{eq:TMAaprox} exact. The methods outlined in \cite{li19963} and \cite{pilkington19973} assume that not only is there no self-demagnetization but also that there is no \ac{NRM}. A slight generalization from assuming no \ac{NRM} is that the anomalous magnetization is entirely in one direction \cite{li19963}. The recovered quantity is effective susceptibility, or magnetization normalized by the earth's field,
\begin{equation} \label{eq:effSusc}
\kappa_{eff} =  \frac{\|\mathbf M\|}{\|\mathbf H\|}.
\end{equation}In the case that the magnitude of \ac{NRM} is negligible, effective susceptibility and the true susceptibility are equal (i.e. $Q \ll 1)$.

	In the context of high remanent magnetization, the assumption that there is no \ac{NRM} is by definition clearly false and \ac{NRM} affects the inversion results. In the case that the \ac{NRM} is in a similar direction as the earth's field, the measured field $B_{TMA}$ will be higher than expected, given only susceptibility, and thus the recovered $\kappa_{eff}$ will be higher than the true $\kappa$. Similarly, in the case that \ac{NRM} is in a direction nearly anti-parallel to the earth's field, the measured field $B_{TMA}$ will be lower than expected, given only susceptibility, and thus the recovered $\kappa_{eff}$ will also be lower than the true $\kappa$.
	
	Understanding the difference between $\kappa$ and $\kappa_{eff}$ is very important with respect to inserting magnetic petrophysical measurements into an inversion's \ac{MOF}. It is an unfortunate truth that susceptibility is significantly easier to measure than \ac{NRM}. As stated above in \autoref{eq:susc} the permeability of an object is related to its susceptibility. In addition the inductance of a coil is proportional to the permeability inside and around it and thus is dependent on the susceptibility of the material. This change in inductance of a coil allows the precise measurement of susceptibility without contamination by \ac{NRM} \cite{collinson1983methods},\cite{clark1991notes}. The measurement of the inductance of a coil also allows the measurement of susceptibility of a rock without reorienting the sample and without shielding from ambient magnetic fields \cite{collinson1983methods}.
	
On the other hand, \ac{NRM} as opposed to susceptibility is a vector quantity and thus the sample must be reoriented to measure it from different directions, even when only the magnitude of the \ac{NRM} is required. In addition, unless the sample is very highly magnetized and stable, measurement will require shielding from ambient fields.

As can be seen from the above, it is not surprising that n the case of El Poma we have many susceptibility measurements including bore-holes and very few (only two within the areas of interest) measurements of \ac{NRM}. However, if susceptibility is used to constrain an inversion in an area of strong \ac{NRM} the inversion could be constrained to a value much higher or lower than the true effective susceptibility. 

At first order, a potential correction given \ac{NRM} measurements is to use the Koenigsberger ratio of a sample and assume that other samples will have a similar Koenigsberger ratio. It is recognized that this assumption will not be true. That said it is a closer approximation than would otherwise be possible without the sample. Once we have a Koenigsberger ratio and a susceptibility we can determine the magnetization of the sample using \autoref{eq:magnetization} and  \autoref{eq:Koenigsberger} in the form
\begin{equation} \label{eq:Qcorrection}
\textbf{M}_{eff} = \kappa\textbf{H} + \|Q\kappa\textbf{H}\|\hat{\mathbf M}_{NRM},
\end{equation}	
where $Q$ is the Koenigsberger ratio used and $\hat{\mathbf M}_{NRM}$ is the magnetization direction of the sample used. It is important to note that \autoref{eq:Qcorrection} is a vector sum and the direction of $\mathbf{M}_{eff}$ will not be in the direction of either $\mathbf H$ or $\mathbf M_{NRM}$. It is also interesting to note that \autoref{eq:Qcorrection} is more generally true if more samples have \ac{NRM} measurements. If, instead of being a single measurement, we have a more detailed estimate of each sample's Koenigsberger ratio and magnetization direction, we can get a better estimate of $\mathbf M_{eff}$.
	
\subsection{Surface Sample Data}
\label{sec: Surface Sample Data}

\subsection{Geological Maps}
\label{sec: Geological Maps}


\section{Using Multiple Data Types, with Clustering}
\label{sec:Using Multiple Data Types, with Clustering}


\endinput

 Interestingly, the assumption that all magnetizations are in the same direction also assumes that all Koenigsberger ratios are equal.

Any text after an \endinput is ignored.
You could put scraps here or things in progress.


%    2. Main body
% Generally recommended to put each chapter into a separate file
%\include{relatedwork}
%\include{model}
%\include{impl}
%\include{discussion}
%\include{conclusions}

%    3. Notes
%    4. Footnotes

%    5. Bibliography
\begin{singlespace}
\raggedright
\bibliographystyle{abbrvnat}
\bibliography{biblio}
\end{singlespace}

\appendix
%    6. Appendices (including copies of all required UBC Research
%       Ethics Board's Certificates of Approval)
%\include{reb-coa}	% pdfpages is useful here
\chapter{Supporting Materials}

\begin{itemize}
\item
\end{itemize}


\backmatter
%    7. Index
% See the makeindex package: the following page provides a quick overview
% <http://www.image.ufl.edu/help/latex/latex_indexes.shtml>


\end{document}
