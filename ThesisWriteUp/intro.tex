%% The following is a directive for TeXShop to indicate the main file
%%!TEX root = diss.tex

\chapter{Introduction}
\label{ch:Introduction}

\begin{epigraph}
    \emph{If I have seen farther it is by standing on the shoulders of
    Giants.} ---~Sir Isaac Newton (1855)
\end{epigraph}


\section{What problems}
\label{What problems}

Geophysical inversions, specifically potential fields 
include formulation of non-regularized inverse problem

\section{Difficulties with said problems }
\label{Difficulties with said problems }

The standard way to fit a set of parameters to a set of data (especialy when they are related by a linear operator) is least squares optimization. This is redered problematic since n general geo-physical inversions are ill-conditioned (define) and underdertermined (define) (Oldenburg and Li 2005 other sources I'm sure). In specific potential fields are particularly under-determined due to the lack of any depth information in the data.

show some form of problems with forward operator matrix in PF inversion

\section{Solutions to said dificultlies}
\label{Solutions to said dificultlies}

To mitigate the difficulties presented above an extra term is added to the optimization. 

\begin{align}
\phi = \phi_d + \beta\phi_m
\end{align}

where $\phi_m$ is called the model objective function. This $\phi_m$ can be defined in many ways, following  (Oldenburg and Li 2005)

\begin{align}
\phi_m(m) &= \alpha_s\int(m-m_{ref})^2dx+\alpha_x\int\bigg(\frac{d}{dx}(m-m_{ref})\bigg)^2dx\\
&=\alpha_s\|\textbf{W}_s(m-m_{ref})\|^2+\alpha_x\|\textbf{W}_x(m-m_{ref})\|^2
\end{align}

in higher dimmesions more smootheness terms can be added. The $\textbf{W}$ terms contain both the operator (indentity for $\textbf{W}_s$ and derivative for $\textbf{W}_x$ and other dimensions) and the relative weight each cell or face contributes to the model norm. This gives us several levers to add a-priori information into the inversion.

\subsection{$\alpha$ coefficients}
\label{alpha coefficients}

broad strokes weights the relative importance of the smallness and smoothness in the various dirrections. can also be thought of as length scales

\subsection{Reference Models}
\label{Reference Models}

we don't always want a model to be close to zero. Sometimes it should be close to a nother constant sometimes we have guesses of the property in some places and want the inversion result to be close to that value

\subsection{Weighting matrices}
\label{Weighting matrices}

much more precise. Can put interfaces in precise locations. Can also force a model towards the reference model where we are more sure

\subsection{Reference Models}
\label{Reference Models}

\endinput

Any text after an \endinput is ignored.
You could put scraps here or things in progress.
