%% The following is a directive for TeXShop to indicate the main file
%%!TEX root = diss.tex

\chapter{Case Study \#2 TKC}
\label{ch:CaseStudy1}

In all cases these are first guesses at what needs to be in each section more or less detail need to be added.



\section{Overview of Deposits}
\label{sec:Overview of Deposits:TKC}

Kimberlite Complex
Two anomalies, focusing on the southern one (DO27)

Magnetic Anomaly. Remanent magnetization is likely present but largely in the dirrection of the earth's field
Density Anomaly. 
\section{Discussion of the Geophysical Data Given}
\label{sec:Discussion of the Geophysical Data Given:TKC}

Magnetics: Three different surveys\\
Gravity: Ground mag (of usable but dubious quality), Gravity Gradiometry airborne data

much EM as well, outside the scope of this Master's Thesis

\section{What Information is Available}
\label{sec:What Information is Available:TKC}

Great deal of borehole data with facies at each depth
We also have Phys Props at various points along the holes. We can either mean these across the facies or take the value of each facies that the specially nearest the the measured result.

From the borehole data we also have created a surface model of each of the units

again from the borehole data, we have graphical cross section maps

\section{Synthetic Model}
\label{sec:Synthetic Model:TKC}

TODO: Create Model
- Already mostly done
- we have a surface from the borehole data and we can use the parametric inversion to assign the properties.
show model
discuss its creation
- magnetization direction

show its fit to the field data

\section{Blind Inversion of the Synthetic Model}
\label{sec:Blind Inversion of the Synthetic Model:TKC}

Show results. show how model is insufficiently compact and over estimates the amount of kimberlite

\section{Determination of Magnetization Dirrection}
\label{sec:Determination of Magnetization Dirrection}

Correlation of Vertical and Total Gradients of Half RTP field \cite{dannemiller2006MagDirection}

taking core direction from MVI result 

apply recovered direction to the anomaly direction in MAG3D
could also use parametric inversion and MVI sensitivities to provide more constraint.

could apply anomalous dirrection locally to anomaly 

In any case the result will be very similar to the earth's field in the location

\section{Creation of Constraints and Types of Data}
\label{sec:Creation of Constraints:TKC}

Extensive Boreholes with rock units
Multiple cross sections (crreated from said bore holes)
Multiple data types to cluster



\subsection{$\alpha$ coefficients}
\label{sec:alpha coefficients:TKC}

not much with alphas to be done here given that we don't expect discontinuity in any one particular dirrection.

\subsection{Reference Models}
\label{sec:Reference Models:TKC}

Most work to be done here. 

We can create a reference model from the phys prop results from the borehole data. Perhaps we should only use some of the boreholes so that we have a more realistic amount of information than in a fully drilled example. We have two ways of applying phys prop measures to inversion and might use both. Also using $K_n$s to improve degree of fit between phys prop measures and effective susc recovered properties
(show reference model)
(show result)

with sufficient boreholes we could make a incorrect surface that approximates the ``true'' model used. Use this with parametric inversion for reference model
(show reference model)
(show result)

using clustering between density and mag (and conductivity and chargeablity) to create clusters, populate each cluster with a value either the mean value of the cluster or a parametric inversion and use as reference
(show reference model)
(show result)

use cross section from \cite{harder2006geologyTKC}
perhaps extend away from line and down weight
(show reference model)
(show result)

(show Combined result)

\subsection{Weighting matrices}
\label{sec:Weighting matrices:TKC}

smallness: using some measure of confidence in the measures decrease in cells with reference model specified to force the result to approximate the reference model. In case the cross section is extended down I will lower the $W_s$ as model cells are further away from the cross section
(show smallness weight model)
(show result, compare to result without)

smoothness: to spread the model values away from where they are specified I can increase the smoothness weights in the vicinity of cells with specified reference models. 
(show face weight model)

with sufficient boreholes we could make a incorrect surface that approximates the ``true'' model used. Use this with parametric inversion for reference mode,l put lower weights along this surface.
(show face weight model)

using clustering between density and mag (and conductivity and chargeablity) to create clusters, populate each cluster with a value either the mean value of the cluster or a parametric inversion and use as reference
(show result)

(show result, compare to result without)

(show Combined result)
\subsection{Bounds}
\label{sec:Bounds:TKC}

Also useful for forcing model values to be near the specified reference model while allowing for uncertainty in our phys prop value. Since we have more statistical info on the 
(show result)

Need to determine if showing the field example is worthwhile at this point and how to bring it into the narrative

\endinput

Any text after an \endinput is ignored.
You could put scraps here or things in progress.
