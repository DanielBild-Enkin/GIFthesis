%% The following is a directive for TeXShop to indicate the main file
%%!TEX root = diss.tex

\chapter{Solutions to Including Information Into \ac{MOF}}
\label{ch:Introduction}

\begin{epigraph}

\end{epigraph}

As Far I as I can tell This section should organized by data type as well.

\section{Forms of A Priori Information}
\label{sec:Forms of A Priori Information}

\subsection{Bore Hole Data and the Use of Koenigsberger Ratios to Correct Bore Hole Susceptibility Measurements}
\label{sec: Bore Hole Data}


I think there is some use in describing how GIFtools does it now. Much of this work was done in \cite{williams2008geologically}. Mostly what I have done is the inclusion of lithologies and the use of Koenigsberger
	
\subsection{Surface Sample Data}
\label{sec: Surface Sample Data}

Again \cite{williams2008geologically} did much of this. I think the Surface Samples will be promary use in proving physical properties for the map in El Poma


\subsection{Geological Maps}
\label{sec: Geological Maps}

This is novel as far as I can tell. I will describe the map to model software and I will describe the use of the ``add face weights along line'' tool to add faults in


\section{Using Multiple Data Types, with Clustering}
\label{sec:Using Multiple Data Types, with Clustering}

There is interesting things to discuss in the storing of multiple inversion in GIFtools, and in the use of clustering algorythms used and then ability to take geological models and make reference models and non-trivial face weighting.

\endinput

 Interestingly, the assumption that all magnetizations are in the same direction also assumes that all Koenigsberger ratios are equal.

Any text after an \endinput is ignored.
You could put scraps here or things in progress.
