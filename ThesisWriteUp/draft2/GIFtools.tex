%% The following is a directive for TeXShop to indicate the main file
%%!TEX root = diss.tex

\chapter{Tools for Integrating Geological and Petrophysical Information into the Regularization of Inversions}
\label{ch:GIFtools}

\section{Including Bore Hole Information in Inversion Regularization}
\label{sec:BHandSS}

Bore holes provide physical property measurements at depth either by providing a conduit to send geophysical instruments down hole or by recovering a core and then measuring its petrophysical properties subsequently in the lab.  The inversion methods for including physical property and lithology bore hole data  have been developed (e.g., \cite{williams2008geologically}). In this section I discuss how GIFtools and Model Builder incorporate bore hole information into inversion constraints in a expedient and effective manner.

In addition to physical property data, bore holes can also provide qualitative rock unit information. In this section I show the linking of lithology information from bore holes with physical property information using petrophysical measurements of reasonably similar rocks. Once this linking is done, the bore hole information can be used in the same fashion as a physical property bore hole logs.

In this section I also show the advantage of integrating the creation of inversion constraints with more general data processing tools. Once we have a set of sample data from either bore hole or surface samples loaded into GIFtools, we can then use the general data modification tools that are already provided for data quality control created both by me and the rest of the \ac{GIF} team.

\subsection{Importing Bore Hole Data}
\label{subsec:importBH}

Bore hole data is typically stored in three separate files: the collar file, the survey file, and the property file. The collar file contains the spacial information of the very top of the hole.  \autoref{tab:collarEx} shows an example collar file. GIFtools is sufficiently flexible that it can read both whitespace and comma delimited files. Additionally as long as there are headers for each column, it does not matter what order the columns appear in, as the correct columns for each purpose can be specified during the import process. Finally while the example shows a text hole identifier, it is also possible to identify individual holes by a numerical index.
\begin{fileExample}
\begin{tabular}{|ccccc|}   
\hline
\multicolumn{5}{|l|}{\% TKC collar data} \\
HOLE-ID & X & Y & Z & LENGTH \\
DO27-05-01 & 557187 & 7133758 & 418 & 58.52 \\
DO27-05-02 & 557191 & 7133755 & 418 & 459.5 \\
DO27-05-03 & 557165 & 7133682 & 418 & 230 \\
DO27-05-04 & 557425 & 7133835 & 420 & 112.5 \\
DO27-05-05 & 557425 & 7133835 & 420 & 99.8 \\
DO27-05-06 & 557425 & 7133835 & 420 & 101 \\
DO27-05-07 & 557425 & 7133835 & 420 & 218 \\
DO27-05-08 & 557392 & 7133834 & 419 & 290 \\
DO27-05-09 & 557392 & 7133834 & 419 & 155 \\
DO27-05-10 & 557392 & 7133834 & 419 & 140 \\
DO27-05-11 & 557400 & 7133913 & 419 & 374 \\
DO27-05-12 & 557345 & 7134210 & 419 & 65 \\
\hline
\end{tabular}
\caption{An example ``collar file'' from TKC bore holes. X and Y are UTM Easting and Northing, Z is the elevation, and Length is the bore hole total length.  All units in meters}
\label{tab:collarEx}
\end{fileExample}

The survey file provides depth, azimuth, and dip information, coding how the hole changes direction below the collar. In \autoref{tab:surveyEx} all the holes as defined in  \autoref{tab:collarEx} are straight and dip in different directions. 
\begin{fileExample}
\begin{tabular}{|cccc|}
\hline
\multicolumn{4}{|l|}{\% TKC survey data} \\
HID & DEPTH & AZIMUTH & DIP \\
DO27-05-01 & 0 & 0 & -90 \\
DO27-05-01 & 58.52 & 0 & -90 \\
DO27-05-02 & 0 & 0 & -90 \\
DO27-05-02 & 459.5 & 0 & -90 \\
DO27-05-03 & 0. & 0 & -90 \\
DO27-05-03 & 230 & 0 & -90 \\
DO27-05-04 & 0 & 180 & -70 \\
DO27-05-04 & 112.5 & 180 & -70 \\
DO27-05-05 & 0 & 200 & -47 \\
DO27-05-05 & 99.8 & 200 & -47 \\
DO27-05-06 & 0 & 80 & -45 \\
DO27-05-06 & 101 & 80 & -45 \\
DO27-05-07 & 0 & 273 & -70 \\
DO27-05-07 & 218 & 273 & -70 \\
DO27-05-08 & 0 & 265 & -45 \\
DO27-05-08 & 290 & 265 & -45 \\
DO27-05-09 & 0 & 265 & -86 \\
DO27-05-09 & 155 & 265 & -86 \\
DO27-05-10 & 0 & 348 & -45 \\
DO27-05-10 & 140 & 348 & -45 \\
DO27-05-11 & 0 & 240 & -45 \\
DO27-05-11 & 374 & 240 & -45 \\
DO27-05-12 & 0 & 230 & -45 \\
DO27-05-12 & 65 & 230 & -45 \\
\hline
\end{tabular}
\caption{An example ``survey file'' from TKC bore holes. Depth is the position along the hole where the change in direction occurs (in meters), Azimuth and Dip are the new direction at the depth provided (degrees). The holes are the same as \autoref{tab:collarEx}}
\label{tab:surveyEx}
\end{fileExample}

Finally, the property file contains information about a given property down the hole. This property can either be a physical property (e.g. density, susceptibility, etc.) or a geological unit. The depth information of a property measurement can be stored as a simple depth along the bore hole, or as an interval with two depths a ``from'' and a ``to'' depth stating that a given measurement hold for the whole interval. \autoref{tab:propEx} shows the second form of property file with a numerical lithology.

\begin{fileExample}
\begin{tabular}{|cccc|}
\hline
\multicolumn{4}{|l|}{\% TKC property data} \\
HOLE-ID & FROM & TO & LITHO  \\
DO27-05-01 & 56.5 & 58.52 & Kimb-1 \\
DO27-05-02 & 56 & 459.5 & Kimb-1 \\
DO27-05-03 & 59 & 230 & Kimb-1 \\
DO27-05-04 & 19 & 63.4 & Kimb-1 \\
DO27-05-04 & 63.4 & 112.5 & Kimb-3 \\
DO27-05-05 & 21.8 & 85.8 & Kimb-1 \\
DO27-05-06 & 37 & 49.5 & Kimb-1 \\
DO27-05-06 & 49.5 & 82.9 & Kimb-3 \\
DO27-05-07 & 20.5 & 104.5 & Kimb-1 \\
DO27-05-07 & 104.5 & 131 & Kimb-3 \\
DO27-05-07 & 138.7 & 218 & Kimb-2 \\
DO27-05-08 & 20.8 & 290 & Kimb-1 \\
DO27-05-09 & 9 & 95.8 & Kimb-1 \\
DO27-05-09 & 95.8 & 117 & Kimb-3 \\
DO27-05-09 & 125.4 & 155 & Kimb-2 \\
DO27-05-10 & 17 & 100.3 & Kimb-1 \\
DO27-05-10 & 100.3 & 123 & Kimb-3 \\
DO27-05-11 & 44.5 & 223.5 & Kimb-1 \\
DO27-05-12 & 36 & 36.7 & Kimb-2 \\
\hline
\end{tabular}
\caption{An example ``property file'' from TKC bore holes. From and To are depth along the hole in meters and Litho is in this case a lithology unit. Other properties can be included in appropriate units. The holes are the same as \autoref{tab:collarEx}}
\label{tab:propEx}
\end{fileExample}

 The process to load bore hole data into GIFtools is as follows. Firstly the files that define the bore hole data set (collar, survey, and property) need to be provided. Secondly, the columns to be imported from the property file need to be stated. Lastly the method by which the property data is linked spatially to the drilling data (collar and survey) is stated. All of these are done by the GUI as shown in \autoref{fig:BHimport1}. Note that the user has to set whether the depth information of the property is based on the sample location (discrete) or each point along the hole is interpolated from the data (interpolate).

 \begin{figure} [h]
    \centering
    \frame{\includegraphics[width=0.8\textwidth]{images/BHSS/BHimport1.PNG}}
    \caption{The first \ac{GUI} for importing bore hole data}
    \label{fig:BHimport1}
\end{figure}

There are two methods to link property and drilling data,  ``Discrete'' and ``Interpolate''. If the ``Discrete'' option is selected then GIFtools simply determined the spatial location based on the depth provided in the property file. In the case that the depth is given in the form of intervals, the depth of the measurement is considered to be the midpoint of the interval. 

If ``Interpolate'' is selected the program behaves differently depending on whether the depth information is simple depths or intervals. In both cases instead of using the depth information directly, a sample is provided at each point along the bore hole with distances between each sample defined by the sampling interval value in the first \ac{GUI} (\autoref{fig:BHimport1}).

In the case of simple depths, it is assumed that the measurement is some physical property, and given the sample depths the measurements are linearly interpolated along the bore hole. In the case of intervals it is assumed that the measurement lithologies should not be interpolated, so samples with depths that are within an interval are assigned the property given while depths outside of any range are assigned a NaN (not a number) value.

Once the files, and property location options have been set, the next step to importing bore hole data into GIFtools is the set of data columns and data headers. Since different collar, survey, and property files have different column orders it advantageous to allow the flexibility to assign columns with a given header to any property of the bore hole object in GIFtools. The \ac{GUI} to assign the columns is shown in \autoref{fig:BHimport2}

 \begin{figure} [h]
    \centering
    \frame{\includegraphics[width=0.8\textwidth]{images/BHSS/BHimport2.PNG}}
    \caption{The file header \ac{GUI} for importing bore hole data}
    \label{fig:BHimport2}
\end{figure}

\subsection{Visualizing Bore Hole Data}
\label{subsec:visBH}

As stated in the introduction to this section, one of the advantages of integrating the Model Builder tools (including these tools for bore hole data being described) into GIFtools, a more general data visualization and quality control environment, is the ability to use those same tools for the data with which we create regularizations. In this case it is very useful to be able to visualize bore hole data before it gets incorporated in reference model and bounds of an inversion. 

\autoref{fig:TKCBHvis} shows the visualization of the bore hole data from TKC. Note that the data has been sampled within each unit.  \autoref{fig:EPBHvis} shows the visualization of a physical property bore hole data set from El Poma. Note that instead of the method used in \autoref{fig:EPBHvis}, each datum along the hole was given its own position, in this case, the mid point of each small interval provided. In both cases the value of the visualized property of a given hole is shown along the side of the 3D representation of all the holes.

 \begin{figure} [h]
    \centering
    \frame{\includegraphics[width=0.8\textwidth]{images/BHSS/TKCBHvis.PNG}}
    \caption{Visualization of TKC lithology bore hole data}
    \label{fig:TKCBHvis}
\end{figure}

 \begin{figure} [h]
    \centering
    \frame{\includegraphics[width=0.8\textwidth]{images/BHSS/EPBHvis.PNG}}
    \caption{Visualization of El Poma susceptibility bore hole data}
    \label{fig:EPBHvis}
\end{figure}

\subsection{Discretizing Bore Hole Data}
\label{subsec:discBH}

Before a bore hole data set can be use in the creation of inversion constraints, it needs to be discretized on the mesh that will be used in the inversion. Discretization allows the multiple data along each bore hole to be usable in the context of an inversion on a given mesh. This will allow the creation of reference models, bounds, and weights. 

The process to discretized bore hole data in GIFtools and Model Builder is as follows.

\begin{itemize}
 \item Provide mesh. If the bore data will be used with a Model Builder module to create constraints, it must have the same mesh as the Model Builder module.
 \item describe distribution 
 \begin{itemize}
  \item normal
  \item log normal
%   \item describe why one and not other. Cond tends to be log normal (citation needed), Den tends to be normal (citation needed), Susc can be either depending on who you ask(citation needed).
 \end{itemize}
 \item describe method of determining bounds
 \begin{itemize}
  \item Confidence interval: given a number of samples in a cell and a distribution, bounds are determined from a given percent confidence interval. Where the standard deviation is zero, (if there is only one datum, or all the data are equal) he minimum value is used in place of the confidence interval.
  \item Floor: Simply assigns a bound based on the mean value of each cell plus or minus the provided floor value
  \item Standard Deviation: Calculates the standard deviation of the sample values in each mesh cell (given a normal or log normal distribution). The bounds are set as equal to the mean value plus some multiple of the standard deviation. In the case that the standard deviation is zero the minimum value is used instead of the multiplied standard deviation. 
 \end{itemize}
 \item positivity simply set the lower bound and mean to be at least 0
\end{itemize}

 \begin{figure} [h]
    \centering
    \frame{\includegraphics[width=0.8\textwidth]{images/BHSS/disc.PNG}}
    \caption{\ac{GUI} to allow the discretization of bore hole data to a given mesh}
    \label{fig:disc}
\end{figure}

\autoref{fig:disc} shows the GIFtools \ac{GUI} that allows the user to input the discretization options.

\subsection{Linking Lithology Information into Petrophysical Information}
\label{subsec:lithBH}

As stated in the introduction of this section, bore hole data often consists of lithology data instead of a physical property. In these cases the lithology information needs to be converted into petrophysical information so that the bore holes are to be used to constrain inversion results. In GIFtools, linking lithology information to petrophysical data is done by what is called a geology definition. The geology definition is a lookup table that contains information of each particular geological unit's property, lower and upper bounds, and optionally the smallness weight associated with each unit. 

Using the geology definition we can convert a geology model that has information about the spatial distribution of geological units but not of their physical properties into constraints that are usable by an inversion. \autoref{fig:geoDefBH} is an example of a geology definition in the GIFtools GUI. The data in the geological definition came from lab measurements of magnetic susceptibility and remancence within each geological unit.

 \begin{figure} [h]
    \centering
    \frame{\includegraphics[width=0.8\textwidth]{images/BHSS/geoDefBH.PNG}}
    \caption{The geology definition of the lithological units in the TKC bore hole data. Effective susceptibility is measured in SI is (in this case)}
    \label{fig:geoDefBH}
\end{figure}

Once the geology definition is set the lithological bore hole can be treated as a physical property bore hole data set where each sample of a given unit is assigned the property in the geology definition. Multiple properties can be stored in the geology definition and the one that will be used to create a set of constraints can be changed by editing the geology definition's I/O header.

\subsection{Editing Bore Hole Data}
\label{subsec:visBH}

\begin{itemize}
 \item show editing of BH discretization
 \item show editing of prop data for susc to eff susc
\end{itemize}

\subsection{Making Constraint}
\label{subsec:makeConstBH}

\begin{itemize}
 \item show resolve conflicts dialogue
 \begin{itemize}
  \item steal from below section (new images necessary)
 \end{itemize}
\end{itemize}

\subsection{Including Surface Sample Information in Inversion Regularization}
\label{subsec:SS}

In addition to physical property measurements down hole, physical properties can also be measured on the surface, in many cases with much more ease. As with bore hole data, much work  has been done on including surface sample information in inversion regularization. Surface samples take on increased importance since in forward models the sensitivity of the data on a given cell is higher when the cell is nearer the surface. Constraining these surface cells can reduce artifacts that come from this increased sensitivity.

\begin{itemize}
 \item importing SS
 \item visualize
 \item editing 
 \item creation of constraints
\end{itemize}

\section{Including Geological Maps in Inversion Regularization}
\label{sec:maps}

It is often the case that geological information is provided in the form of geological maps in either cross section or plan view. Such maps are particularly useful since they provide a great deal of information over their entire surface. Cross sections can provide information at depth and constrain a whole region often within the center of a target of interest. Plan view maps do not provide information at depth, but they do constrain the entire surface of the region being inverted.  Constraining the surface of an inversion is of interest since the sensitivity of the data to the top cells is particularly high, which can lead to artifacts on the surface. For the next section the plan view model is from the El Poma case study and the cross section model is from TKC, specifically the map from \citep{harder2006geology}

\cite{williams2008geologically} discuses methods to include maps in the form of ESRI shapefiles. His method has the disadvantage of not being able to incorporate information from maps stored a pixel images. On the other hand, a method that allows the incorporation of pixel images allows the use of ESRI shapefiles since the conversion of a shapefile to a pixel image is trivial.

Below is the method I have developed to incorporate pixel maps.

\subsection{ Preprocessing images}
\label{subsec:Preprocessing images}

Often a geological map image will not be immediately suitable to the methods used below and some prepossessing is required. The most notable features that are undesirable in a map are geological units that are not only one or two colours and text or other annotations that could be interpreted by the program as geological information. Also map images may be of too high resolution to be efficiently used in these methods and must also be down-sampled to save  computer processing time and memory.

 \begin{figure} [h]
    \centering
    \frame{\includegraphics[width=0.8\textwidth]{images/Faults/ElPoma_GEOL_MagSus.png}}
    \caption{The El Poma map with fault lines (blue lines with barbs) included}
    \label{fig:ElPoma_GEOL_MagSus}
\end{figure}

For example \autoref{fig:ElPoma_GEOL_MagSus} has great deal of information, (faults, magnetic susceptibility surface samples, etc.) that are not information about geological units. In addition the geological units are not a single colour polygon. The image has been edited in the GNU Image Manipulation Program (GIMP), a free image editing program, to produce \autoref{fig:ElPoma_GEOL_MagSus_M2M_downsample8}. 

 \begin{figure} [h]
    \centering
    \frame{\includegraphics[width=0.8\textwidth]{images/MaptoModel/ElPoma_GEOL_MagSus_M2M_downsample8.png}}
    \caption{The El Poma map with extra information removed and geological units made a single colour}
    \label{fig:ElPoma_GEOL_MagSus_M2M_downsample8}
\end{figure}

\FloatBarrier
\subsection{Loading Images into GIFtools}
\label{subsec:Load Images into GIFtools}

\begin{itemize}
\item load image into the GIFtools format (\autoref{fig:importPlanGui})
\begin{itemize}
	\item Determine image format.
	\item Load image using MATLAB utilities.
	\item Convert image into .png style representation for faster computation.
	\item Using .twf file (world file) assign location and spacial resolution to the image.
	\item Assign a legend linking pixel RGB values to geological unit.
	\item Assign topography (either number or GIFtools TOPOdata item) for visualization.
	\begin{itemize}
		\item In the case of a cross section image, instead of topography, information for the location of the cross section in 3D or 2D space is required (\autoref{fig:importCrossGui}).
	\end{itemize}
\end{itemize}
\end{itemize}
\begin{figure} [h]
    \centering
    \includegraphics[width=0.5\textwidth]{images/MaptoModel/importPlan.PNG}
    \caption{\ac{GUI} for importing plan view image }
    \label{fig:importPlanGui}
\end{figure}
\begin{figure} [h]
    \centering
    \includegraphics[width=0.5\textwidth]{images/MaptoModel/importCross.PNG}
    \caption{\ac{GUI} for importing cross section image }
    \label{fig:importCrossGui}
\end{figure}
Storing a map as a GIFtools object allows its use in several ways. Notably it allows the integration of the map with models and data, allowing figures overlaying the map and data or model and allowing interpretation of the data or model with direct reference to the map (\autoref{fig:mapData}).
\begin{figure} [h]
    \centering
    \includegraphics[width=0.5\textwidth]{images/MaptoModel/mapData.PNG}
    \caption{Example of magnetics data being viewed with a map overlaid }
    \label{fig:mapData}
\end{figure}



\subsection{Creating a Pixel Map Legend}
\label{subsec:Create Pixel Map Legend}

Continuing on in the process of making a geological constraint.
\begin{itemize}
\item Find the geological unit represented of each pixel.
\begin{itemize}
	\item In the .png style format as stored in MATLAB, an image consists of an ``image'' field, a matrix of integers, and a ``map'' field, which maps the image matrix to RBG value triplets.
	\item Each RGB triplet is compared to the legend that was provided when the image was loaded. A map field entry is considered to represent a geological unit if all three components of the RGB triplet are within a provided tolerance of any entry in the legend.
	\item Now that we have a relation of entries in the map field to geological units in the legend, we can assign a geological unit to each pixel in the original image simply by applying the new geological map to the image field.
\end{itemize}
\end{itemize}

\subsection{Making a Geology Model from Map}
\label{subsec:Make Geology Model from Map}

\subsubsection{Plan View}
\label{subsubsec:Make Geology Model from Map Plan View}

\begin{itemize}
\item Provide active model. For convenience this is usually an active model already associated with a Model Builder object.
\begin{itemize}
	\item The active model simultaneously provides a discretized topography for the map to lay along and also a mesh (\ac{GIF} 3D tensor or OcTree).
\end{itemize}
\item Provide some form of depth information.
\begin{itemize}
	\item Thickness, a certain amount of depth below topography at each point will be assigned the geological unit at each.
	\item Depth, the map will be used to assign a geological unit down to a fixed depth across the whole model.
	\item Surface, if you provide another surface below topography the cells between topography and the other surface will be assigned.
\end{itemize}

\item Crop all pixels that extend outside of the mesh or that represent the background geological unit.
\begin{itemize}
	\item The cropping greatly speeds up the process and makes it require much less computer memory.
	\item Furthermore, in the event of a mistake with coordinates the process ends almost instantly as there are few pixels to process.
\end{itemize}
\item Finally the geological model is created.
\begin{itemize}
	\item We determine which cell of the mesh each pixel is in, including those cells below each pixel to account for thickness.
	\item Each cell is assigned a geological unit based on the mode of the geological values of each pixel which colours that cell.  In other words, each cell is identified with the geological unit which fills the greatest proportion of the cell.
	\begin{itemize}
	\item The mode is used since each cell will be a particular unit. Since the property being mapped onto each cell by construction must represent a single geological unit, interpolation between the units will not provide the desired result.
	\end{itemize}
	\item The geology definition which will allow the assignment of physical properties to each geological unit. The result is shown in \autoref{fig:mapModelPlan}, the continuous colour bar is not an indication of a continuous model. All model values are integers that represent geological units in the map.
\end{itemize}
\end{itemize}
\begin{figure} [h]
    \centering
    \includegraphics[width=0.5\textwidth]{images/MaptoModel/mapModelPlan.PNG}
    \caption{Example of a geology model created from a map with the map overlaid}
    \label{fig:mapModelPlan}
\end{figure}

\subsubsection{Cross Section}
\label{subsubsec:Make Geology Model from Map Cross Section}

The cross section case follows much the same procedure with a few exceptions. An imported cross section map is shown overlaid on a 2D mesh in \autoref{fig:mapMeshCross}. Notably no parameter for the vertical extent is needed. The other notable exception is that mesh that is used is a \ac{GIF} 2D mesh. The result is shown in \autoref{fig:mapModelCross}.  

A 2D Geology model can be used to create constraints for a 2D inversion, it can also be used to add constraints to a 3D inversion as well. After the 2D geology model is created from the cross section map, it can be inserted into a 3D mesh (\ac{GIF} 3D tensor or OcTree) given a starting and ending position or a starting position and a direction \autoref{fig:add2Dto3D},\autoref{fig:mapModelCross3D}.

\begin{figure} [h]
    \centering
    \includegraphics[width=0.5\textwidth]{images/MaptoModel/mapMeshCross.PNG}
    \caption{Example of a 2D mesh with the map overlaid}
    \label{fig:mapMeshCross}
\end{figure}
\begin{figure} [h]
    \centering
    \includegraphics[width=0.5\textwidth]{images/MaptoModel/mapModelCross.PNG}
    \caption{Example of a 2D geology model created from a cross section map with the map overlaid}
    \label{fig:mapModelCross}
\end{figure}
\begin{figure} [h]
    \centering
    \includegraphics[width=0.5\textwidth]{images/MaptoModel/add2Dto3D.PNG}
    \caption{GUI for adding a 2D model to a 3D model}
    \label{fig:add2Dto3D}
\end{figure}
\begin{figure} [h]
    \centering
    \includegraphics[width=0.5\textwidth]{images/MaptoModel/mapModelCross3D.PNG}
    \caption{Example of a 2D geology inserted into a 3D model with the map overlaid}
    \label{fig:mapModelCross3D}
\end{figure}
\FloatBarrier
\subsection{Making Constraints for an Inversion}
\label{subsec:Making Constraints for an Inversion}

The model that has been created is a geology model. That is, a model in which each cell represents a given geological unit. To be able to convert this model into a constraint for a geophysical inversion the link between between the geology and the petrophysics needs to be provided. 

The link is stored in what is called a geology definition. In GIFtools this takes the form of a lookup table that contains information of each particular geological unit's property, lower and upper bounds, and optionally the smallness weight associated with each unit. 

Using the geology definition we can convert a geology model that has information about the spatial distribution of geological units but not of their physical properties into constraints that are usable by an inversion. In the figures below the geological definition came from surface measurements of magnetic susceptibility within each geological unit. \autoref{fig:geoDefPlan} is an example of a geology definition in the GIFtools GUI.
\begin{figure} [h]
    \centering
    \includegraphics[width=0.8\textwidth]{images/MaptoModel/geoDefPlan.PNG}
    \caption{Example of a geological definition as displayed in the GIFtools GUI, the Mean Property, and Bounds are susceptibility measurement in SI$\times10^-3$ and the WS is to set the smallness weight model.}
    \label{fig:geoDefPlan}
\end{figure}

Once the geology definition is provided, we can use the Combine Model Dialog (\autoref{fig:combineModelRef}) in Model Builder to create a reference model and bounds. 
\begin{figure} [h]
    \centering
    \includegraphics[width=0.8\textwidth]{images/MaptoModel/combineModelRef.PNG}
    \caption{Example of a typical combine model dialog for a reference model}
    \label{fig:combineModelRef}
\end{figure}
In this case the resolution of conflicts is trivial as there is a single source of information. Less trivial examples of the creation of reference models and bounds will be discussed later. The resulting reference model is shown in \autoref{fig:mapRefModPlan}.
\begin{figure} [h]
    \centering
    \includegraphics[width=0.8\textwidth]{images/MaptoModel/mapRefModPlan.PNG}
    \caption{Example of a reference model created from a geological map}
    \label{fig:mapRefModPlan}
\end{figure}

\subsection{Inputing Fault information from Geological Maps}
\label{subsec:Inputing Fault information from Geological Maps}

Another piece of information that can be in geological maps are fault locations. Again in the context of El Poma the map provided a whole complex of thrust faults as shown in the un-doctored map in \autoref{fig:ElPoma_GEOL_MagSus}

The method used to insert faults into an inversion is as follows:

\begin{itemize}
\item Determine the end points of the fault.
\begin{itemize}
	\item GIFtools makes this easy by reporting the location of the cursor in the data viewer allowing you to find the location (including elevation) of a point along the fault.
\end{itemize}
\item Using the locations provided GIFtools creates the fault weights by creating a two boxes, each with one of its sides along the fault location as defined in the \ac{GUI}. By setting the value of cells in each box to one, and then taking the derivative of each of models that had a box added, two face models are created. The location of faces along the fault can be determined by taking the non-zero faces that are in common between the two models.  It is then possible to set the values of the the faces that define the fault to any value desired.

\item faces within this box are assigned a new value that is provided in the GUI \autoref{fig:makeFaultGUI}.
\end{itemize}

\begin{figure} [h]
    \centering
    \frame{\includegraphics[width=0.5\textwidth]{images/Faults/makeFaultGUI.PNG}}
    \caption{The GUI for the creation of fault weights}
    \label{fig:makeFaultGUI}
\end{figure}

This process can be done multiple times along several consecutive segments of multiple faults to create fault complexes that are not straight of have multiple faults, as shown in \autoref{fig:faults}

\begin{figure} [h]
    \centering
    \frame{\includegraphics[width=0.5\textwidth]{images/Faults/faults.png}}
    \caption{An example of fault weights that can be created with GIFtools. In this case a vertical fault (no dip) was created and the blue curtain is the low face weights that define the fault in an inversion. Only the faces that have values that are not 1 are visualized. Since the faces are set at a lower value than the rest of the face model (1) they are shown as blue.}
    \label{fig:faults}
\end{figure}
\FloatBarrier

\section{Clustering to Create Constraints}
\label{sec:cluster}

\subsection{Clustering Algorithms}
\label{subsec:clusterAlgo}

In contexts where multiple data types have been collected over an area, it is often of interest to include information from one inversion result (from one data type) in the regularization of another (likely of a different data type). Much work has been done on this topic (see \autoref{subsec:litrevLitho}). 

One possible way of integrating information into one inversion from another inversion result is through clustering the recovered models to create pseudo-geological models, and then using these to create reference models, bounds, and smoothness weights to constrain subsequent inversions. By pseudo-geologic I mean a geologic type model that has discrete units and a geologic definition that links them to physical propoerties, but which is not the result of a geological interpretation of surface or depth litholgical data. Rather pseudo-geological models in this thesis tend to be the result of clustering multiple inversion results.

Given several inversions of different data types, multiple models of different physical properties can be recovered. Assuming that each recovered model is on the same mesh (either because all inversions were performed on the same mesh or they were interpolated onto the same mesh after the fact) each cell has a value for each physical property. Since the standard representation of a model is a vector $\mathbf m \in \mathbb R^M$ where as in \autoref{eq:forwardProb} $M$ is the number of cells in the discretization of the earth model, we can generalize so that $\mathbf M \in \mathbb R^{M\times l}$ is the compiled results of $l$ inversions. This notation allows us to say that $\mathbf M$ is made out of several $m_i$ for $i = 1 \cdots M$ and where each $m_i$ is a row vector $l$ long.

In each case the algorithm depends on the user providing the number of clusters for analysis. The clustering method used determines the clusters and the membership of each $m_i$ in the clusters.

The first clustering methods implemented in Model Builder is simply a set of user defined boundaries. These allow a great deal of user control and can be very useful assuming something is known about the underground geology or the physical properties of given units. On the other hand, it can quickly be untenable to use this method if several clusters are needed or many recovered models are being clustered.

The second clustering method implemented in Model Builder is k-means clustering \cite{gaf1984multivariate} which is based on the minimization of the following objective function,

\begin{equation}
 \phi_{k\text{-}mean} = \sum_{j=1}^k\sum_{m_i \in S_j}\|m_i - v_j\|_2^2, \label{eq:kmean}
\end{equation}


where $k$ is the number of clusters and $S_j$ are the sets of model cells $m_i$ that make up the clustering, finally $v_j$ are the centroids of each cluster, that is the central tendency of the data included in the given cluster $S_i$. The algorithm to minimize $\phi_{\text{k-mean}}$ is a two step process firstly assigning each $m_i$ to a cluster $S_j$ and secondly determining a new set of $v_j$ that better fit the data.

Finally, the third clustering algorithm implemented in Model Builder is \acf{FCM} \citep{sun2015multidomain}. \ac{FCM} is a generalization of \autoref{eq:kmean} where instead of each datum $m_i$ being in only one cluster, each datum is assigned membership in each cluster to varying degrees allowing ``fuzziness'' in the classification. The objective function becomes

\begin{equation}
 \phi_{FCM} = \sum_{j=1}^k\sum_{i = 1}^Mu^q_{ij}\|m_i - v_j\|_2^2,\label{eq:fcm}
\end{equation}

where instead of the $k$ sets $S_j$, membership in each cluster is represented by the membership matrix $u \in \mathbb R^{M\times k}$. Each row of $u$ must sum to 1, in other words each datum is is evenly weighted in the algorithm but may be classified as partially in each cluster. Finally $q$, controls the ``fuzziness'' of the clustering with a value of $q = 1$ making a non-fuzzy clustering with each datum being in only one cluster and amount that a given datum can be in multiple clusters increasing as the value of $q$ increases. A standard value of $q$ is 2.

In the case of the clustering algorithm used in Model Builder the final step is ``de-fuzzification'', that is the determining of the cluster that each model cell most fits in. Each model cell is assigned the cluster for which it has the highest membership value.

In both the k-means and the \ac{FCM} case the minimization of the objective functions is done by the standard MATLAB function and notably the value of $q$ in the \ac{FCM} case is set at 2.

\subsection{Clustering In GIFtools and Model Builder}
\label{subsec:clusterTools}

\begin{itemize}
 \item describe requirements
 \item show GUI
 \item show result
\end{itemize}

\subsection{Creation of Constraints}
\label{subsec:clusterConstraints}

\begin{itemize}
 \item show creation of ref and bounds from mean cluster boundaries
 \item show creation of sharp bounds with weights.
\end{itemize}

\section{Voxel-Parametric Inversion to Provide Physical Property Values for Geological Models}
\label{sec:voxelParam}

In \autoref{sec:cluster} geological type models that span the whole discretized volume are defined. These are more extensive than the geological models described in \autoref{subsec:Make Geology Model from Map} because each cell is classified into a cluster as opposed to most cells not having any information (due to the map not intersecting the cell at all in the case of most cells).

In addition to pseudo-geological models created by clustering, it is also possible to get geological models from drill or surface geological data. In the TKC case study, one of the forms of geological data that was provided was a set of points created from the interpreted interfaces between units determined using the litholgical bore hole data shown in \autoref{fig:TKCBHvis}. The point cloud of the interfaces is shown in \autoref{fig:geoLoc} and the resultant geological model created using the ``add non-convex polyhedron'' command is shown in \autoref{fig:geoMod}.

\begin{figure} [h]
    \centering
    \frame{\includegraphics[width=0.5\textwidth]{images/BHSS/geoLoc.PNG}}
    \caption{Point cloud of the geological interfaces for TKC bore hole data}
    \label{fig:geoLoc}
\end{figure}

\begin{figure} [h]
    \centering
    \frame{\includegraphics[width=0.5\textwidth]{images/BHSS/geoMod.PNG}}
    \caption{A geological model created from the data shown in \autoref{fig:geoLoc}}
    \label{fig:geoMod}
\end{figure}

In the cases listed in (the figure that shows a clustered model) and \autoref{fig:geoMod}, we have acquired a distribution of geological units that extend across the whole discretized volume it becomes interesting to go further into the assignment of physical properties to the geological units in question. For a clustered pseudo-geological model the obvious method of property assignment is simply using the centroid value recovered from the clustering algorithm. In addition a standard deviation from the mean can be acquired to set the upper and lower bounds of each unit.

Another option is the direct assignment of measured physical property values to the (pseudo-)geological units. This method has the advantage of directly incorporating physical property data into the inversion constraints. Direct assignment can be done through the GIFtools \ac{GUI} by editing the geologic definition of the geological model, and this method makes most sense when the model in question either is the products of geological modeling as in \autoref{fig:geoMod} or there is a strong parallel between the clustered units and actual geological units. 

In some cases the above method are not optimal. In the case of cluster centroids, artifacts in the inversion can lead to magnitudes in the recovered model not being quite in scale with the physical properties in the ground. Additionally this method only works if the geological model in question was derived through clustering. In the case of the direct assignment of measured physical properties to geological units, the properties often vary a great deal across the earth model. This can be due to large scale variation across the geological unit (such as by alteration) or small scale variation where, for example, one portion of a sample can be significantly more susceptible than another.

I have developed another method to assign physical properties to geological models which is well-described by the title ``Voxel-Parametric inversion''. It assumes that a given geological model has correctly determined the distribution of geological units and tries to fit a given geophysical data set assuming that each unit has a single physical property value that is constant across the unit. 

The method allows the assignment of physical properties to geological units based on the best fitting set of properties to a data set. This gives an advantage over the methods described above by avoiding some artifacts due to under-determined inversion and avoiding putting undue confidence in a small number of physical property measurements. The method also allows fast and efficient inversion of data without a large number of parameters since the forward problem is much less complex and regularization beyond the setting of the geological units is unneeded.

\subsection{Formulation of Voxel-Parametric inversion problem}
\label{subsec:voxelParamFormulation}

As stated in \autoref{sec:Regularized Inversion} and \autoref{eq:forwardProb}, the forward problem for which we are trying to find the inverse is formulated as follows
\begin{equation}
\mathbf d = \mathbb F [\mathbf m], \label{eq:forwardProbTools}
\end{equation}
where as stated above $\mathbf d \in \mathbb R^N$ is the geophysical data, $\mathbf m \in \mathbb R^M$ is the discretized model that describes the distribution of some physical property in the ground, and $\mathbb F$ is the forward operator that mediates between them. $\mathbb F$ can be considered to be a matrix of size $N\times M$ which in the case of a non-linear problem has a dependence on the model $\mathbf m$ but in the case of linear problem (such as potential field problems) there is no such dependence. \autoref{eq:forwardProbTools} can be re-written
\begin{equation}
\mathbf d = \mathbf F(\mathbf m)\mathbf m, \label{eq:forwardProbMat}
\end{equation}
where $\mathbf F \in \mathbb R^{N\times M}$, and in the case of a linear problem
\begin{equation}
\mathbf d = \mathbf F\mathbf m. \label{eq:forwardProbMatLin}
\end{equation}

In this formulation a geological model can be represented as a sparse matrix $\mathbf{M}_g \in \mathbb R^{M\times k}$ (where $k$ is the number of geological units) that marks the membership of each model cell in each cluster and a vector $\mathbf m_g \in \mathbb R^{k}$ that has the property of each unit. Using these definitions we can define the standard property model that is represented by a geological model as
\begin{equation}
\mathbf m = \mathbf M_g\mathbf m_g, \label{eq:geoMod}
\end{equation}
where $\mathbf m$ is a standard voxel model with the each value of each cell being the value assigned to geological unit of that cell. \autoref{eq:geoMod} can be inserted intro \autoref{eq:forwardProbMatLin} to form 
\begin{equation}
\mathbf d = \big(\mathbf F\mathbf M_g\big)\mathbf m_g, \label{eq:forwardProbMatLinGeo}
\end{equation}
where $\mathbf F\mathbf M_g \in \mathbb R^{N\times k}$ is created by a sparse matrix product and is significantly smaller than $\mathbf F \in \mathbb R^{N\times M}$.

Voxel-Parametric inversion now simply consists of minimizing the following data objective function modified from \autoref{eq:phid}
\begin{equation}
\phi_{d\text{-}geo} =\|\mathbf W_d\Big(\big(\mathbf F\mathbf M_g\big)\mathbf m_g - \mathbf d^{obs}\Big)\|^2
\end{equation}
\label{eq:phidGeo}
which is significantly less computationally expensive than a full voxel inversion since the forward problem is now much smaller, and no additional regularization is required since the problem is now over determined rather than under-determined.

\subsection{Uses of Voxel-Parametric inversion results}
\label{subsec:voxelParamUses}

 \begin{itemize}
  \item assign properties to geological models that fit the collected data as closely as possible
  \begin{itemize}
   \item allows the geology constraints as in \autoref{subsec:clusterConstraints} to be more accurate
   \item allows the determination of magnetization direction given a geologic or pseudo-geologic model of an magnetization anomaly   
   \item allows the creation of synthetic models that roughly fit data that has already been acquired
  \end{itemize}
 \end{itemize}

\section{Conclusion}
\label{sec:GIFtoolsConc}

In this section I have shown the creation of constraints that are compatible with \ac{GIF} inversion codes. I have created these constraints from multiple types of data: bore hole and surface sample data, maps (Cross Section and Plan View), and other inversion results (clustering and voxel-parametric inversion). I have also  used different pieces of information from these forms of data to create reference models, bounds, and face weights.



%\section{Using Multiple Data Types, with Clustering}
%\label{sec:Using Multiple Data Types, with Clustering}
%
%There is interesting things to discuss in the storing of multiple inversion in GIFtools, and in the use of clustering algorythms used and then ability to take geological models and make reference models and non-trivial face weighting.

\endinput

Any text after an \endinput is ignored.
You could put scraps here or things in progress.
