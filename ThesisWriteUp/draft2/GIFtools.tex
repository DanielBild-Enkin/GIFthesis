%% The following is a directive for TeXShop to indicate the main file
%%!TEX root = diss.tex

\chapter{Solutions to Including Geological Maps Into \ac{MOF}}
\label{ch:GIFtools}
%
%\begin{epigraph}
%
%\end{epigraph}
%
%As Far I as I can tell This section should organized by data type as well.
%
%\section{Forms of A Priori Information}
%\label{sec:Forms of A Priori Information}
%
%\subsection{Bore Hole Data and the Use of Koenigsberger Ratios to Correct Bore Hole Susceptibility Measurements}
%\label{sec: Bore Hole Data}
%
%
%I think there is some use in describing how GIFtools does it now. Much of this work was done in \cite{williams2008geologically}. Mostly what I have done is the inclusion of lithologies and the use of Koenigsberger
%	
%\subsection{Surface Sample Data}
%\label{sec: Surface Sample Data}
%
%Again \cite{williams2008geologically} did much of this. I think the Surface Samples will be promary use in proving physical properties for the map in El Poma
%
%
%\subsection{Geological Maps}
%\label{sec: Geological Maps}

It is often the case that geological information is provided in the form of geological maps either cross section or plan view. These maps are of particular use since they provide a great deal of information over their entire surface. Cross sections can provide a great deal of information at depth and constrain as whole region often within the center of a target of interest. While plan view maps cannot provide information at depth they can constrain the entire surface of an inversion.  Constraining the surface of an inversion is of particular interest since the sensitivity of the data to the top cells is particularly high which can lead to artifacts on the surface.

Below in point form is the method I use to incorporate a pixel map

\begin{itemize}
\item load image into the GIFtools format
\begin{itemize}
	\item determine image format
	\item load image using MATLAB utilities
	\item convert image into .png style representation for faster computation
	\item using .twf file (world file) assign location and spacial resolution to the image
	\item assign a legend assigning pixel RGB values to geological unit
\end{itemize}
\end{itemize}

Storing a map as a GIFtools object allows it's use in several ways. Notably it allows the integration of the map with models and data, allowing figures overlaying the map and data or model and allowing interpretation of the data or model with direct reference to the map.

Continuing on in the process of making a geological constraint, in the plan view case

\begin{itemize}
\item provide active model	
\begin{itemize}
	\item this simultaneously provides a discretized topography for the mesh to lay along and also a mesh
\end{itemize}

\end{itemize}


%\section{Using Multiple Data Types, with Clustering}
%\label{sec:Using Multiple Data Types, with Clustering}
%
%There is interesting things to discuss in the storing of multiple inversion in GIFtools, and in the use of clustering algorythms used and then ability to take geological models and make reference models and non-trivial face weighting.

\endinput

 Interestingly, the assumption that all magnetizations are in the same direction also assumes that all Koenigsberger ratios are equal.

Any text after an \endinput is ignored.
You could put scraps here or things in progress.
