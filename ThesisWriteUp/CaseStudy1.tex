%% The following is a directive for TeXShop to indicate the main file
%%!TEX root = diss.tex

\chapter{Case Study \#1 El Poma South}
\label{ch:CaseStudy1}

In all cases these are first guesses at what needs to be in each section more or less detail need to be added.

\section{General Overview of El Poma}
\label{sec:General Over View of El Poma}

Two anomalies. North and South. This section discusses the southern anomaly. The next discusses the northern one. 

Magnetic Anomaly. Remanent magnetization is clearly present

\section{Overview of Deposits}
\label{sec:Overview of Deposits:ElPoma1}

\section{Discussion of the Geophysical Data Given}
\label{sec:Discussion of the Geophysical Data Given:ElPoma1}

Magnetics. Missing a corner.

\section{What Information is Available}
\label{sec:What Information is Available:ElPoma1}

Bore Hole
-susceptibilities, much lower than the recovered model sue to remanent effects being present

Plan View Geological map
-with susceptibility surface samples marked, in addition to surface samples and geological units, we also have a system of thrust faults over top of both anomalies.

Surface Samples
-susceptibility, same as marked on map but includes many samples from outside map area as well
-also have nine remanences measured with direction and $K_n$

\section{Synthetic Model}
\label{sec:Synthetic Mode:ElPoma1l}

TODO: Create Model
- make iso-surface of Kris's result. Determine property from parametric inversion.
show model
discuss its creation
- magnetization direction

show its fit to the field data

\section{Blind Inversion of the Synthetic Model}
\label{sec:Blind Inversion of the Synthetic Mode1:ElPoma1}

Show results. Discuss how magnetization direction puts anomaly away from actual location

\section{Determination of Magnetization Dirrection}
\label{sec:Determination of Magnetization Dirrection}

Correlation of Vertical and Total Gradients of Half RTP field \cite{dannemiller2006MagDirection}

taking core direction from MVI result 
could also use parametric inversion and MVI sensitivities to provide more constraint.

apply recovered direction to the anomaly direction in MAG3D
could apply anomalous dirrection locally to anomaly 

\section{Creation of Constraints}
\label{sec:Creation of Constraints:ElPoma1}

\subsection{$\alpha$ coefficients}
\label{sec:alpha coefficients:ElPoma1}

For El Espa\~nole (north south fault) we can lower the $\alpha_x$ to allow for greater discontinuity in general in that direction. Cannot account for other faults without rotation objective function.

(show result)

\subsection{Reference Models}
\label{sec:Reference Models:ElPoma1}

Most work to be done here. 

Borehole: provides susceptibilities need to convert into effective susceptibilities. Assuming uniform magnetization direction this is not complicated. Choose a $K_n$ and multiply susceptibility by that (maybe with +1). For MVI I need to apply the dirrection of magnetization as well. Either from the truth of the synth model from direction of the nearest remanent sample or from the bulk rem mag direction.
(show reference model)
(show result)


Map: geological units have susceptibilities attached. Have to convert into effective susceptibilities. Might extend the map cells down below surface to be less weighted
(show reference model)
(show result)

Surface Samples: used to make susc values for geological units. Can also be used for reference model directly but this provides less cover of the surface. Perhaps use surface samples in white region instead of just applying nothing.
(show reference model)
(show combined map and SS reference model)
(show result)

(show Combined result)

\subsection{Weighting matrices}
\label{sec:Weighting matrices:ElPoma1}

smallness: using some measure of confidence in the measures decrease in cells with reference model specified to force the result to approximate the reference model. In case that map is extended down I will lower the $W_s$ as model cells are further below the surface.
(show smallness weight model)
(show result, compare to result without)


smoothness: to spread the model values away from where they are specified I can increase the smoothness weights in the vicinity of cells with specified reference models. 
(show face weight model)
(show result, compare to result without)
The other application of smoothness weights is to allow discontinuities on the faults (perhaps with increased smoothing on either side of the fault). Need to experiment with orientations of the faults.
(show result)

(show Combined result)
\subsection{Bounds}
\label{sec:Bounds:ElPoma1}

Also useful for forcing model values to be near the specified reference model while allowing for uncertainty in our phys prop value
(show result)

\subsection{$L_p L_q$ weights}
\label{sec:Lp Lq weights:ElPoma1}

allows the more fuzzy placement of faults. By rotating the \ac{MOF} we can place them in arbitrary directions. The trouble is having more than one fault in more than one orientation. Can't currently apply to MVI inversions, can still apply on MAG3D inversions 
(show result)
\\\\\\
Need to determine if showing the field example is worthwhile at this point and how to bring it into the narrative

\endinput




Any text after an \endinput is ignored.
You could put scraps here or things in progress.
